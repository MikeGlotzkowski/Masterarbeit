\chapter*{Zusammenfassung}

{\normalsize ZIEL} \par
In der vorliegenden Arbeit werden Einflussfaktoren auf den Kurs von ausgewählten Kryptowährungen (Bitcoin [BTC] und Ethereum [ETH]) gesucht und mit Hilfe von Machine Learning analysiert, ob sich die Kurse der digitalen Währungen voraussagen lassen. Das Machine Learning wird mit dem Werkzeug Azure Machine Learning Studio von Microsoft durchgeführt.

{\normalsize VORGEHEN} \par
Zuerst werden im Theorieteil die Grundlagen (Data Mining Frameworks, Machine Learning, Kryptowährungen und Azure Machine Learning Studio) beschrieben. Hier wird die Wahl getroffen, die Analyse mit dem CRISP-DM-Referenzmodell durchzuführen.\par
Anschließend folgt der Praxisteil. Es werden die ersten fünf Phasen des Referenzmodells durchlaufen. Die letzte Phase (Deployment) entfällt.

{\normalsize GENUTZTE EINFLUSSFAKTOREN} \par
Folgende Einflussfaktoren werden für die Untersuchung genutzt:
\begin{itemize}
\item Kryptowährungseigene Faktoren (Handelsvolumen, Erzeugungsschwierigkeit, Preis am Vortag, Anzahl der Transaktionen etc.)
\item öffentliches Interesse (Google Suchen, Google Newssuchen, Wikipedia Seitenaufrufe, Zeitungsüberschriften)
\item Aktienindizes (aus verschiedenen Wirtschaftsregionen und unterschiedlichen Sektoren)
\item der Preis für natürliche Ressourcen (zwei Ölsorten, Gold, Silber)
\item historische Kursdaten für den \gls{btc}- und \gls{eth}-Kurs
\end{itemize}

{\normalsize ERGEBNISSE DES MACHINE LEARNING} \par
Die Ergebniswerte des Machine Learning befinden sich in den Tabellen \ref{tab:ETH2} bis \ref{tab:BTCReg}.\par
Wie erwartet, lässt sich schlussfolgern, dass der Kurs mit den angewandten Mitteln und gefundenen Einflussfaktoren nicht vorher gesagt werden kann. Es ist jedoch zu erkennen, dass einige Faktoren, wie Google Suchen, einen großen Einfluss haben. Auch wenn Tendenzen im Kursverlauf durch Machine Learning aufgedeckt werden können, gibt es zu viele Einflüsse, über die keine Daten vorhanden sind. So wird beispielsweise geschätzt, dass über 50-80\% der Miningpower im Bitcoinnetzwerk von chinesischen Miningpools kontrolliert wird \multicitep{wagenknecht_bitcoin_2016; tuwiner_10_2017}. Die Absprachen dieser Pools können großen Einfluss haben. Aus diesem Grund wird empfohlen, sich auf wenige Einflussfaktoren fokussieren, diese aber genauer zu betrachten. 
Der Kurs des Bitcoins scheint stärker von Faktoren beeinflusst zu werden, als der des Ethers.

{\normalsize BEWERTUNG DES CRISP-DM-REFERENZMODELLS} \par
\gls{crisp} ist ein sehr vielseitiges Referenzmodell. Es bietet durch den User Guide eine große Hilfestellung für weniger Erfahrene. In leicht angepasster Version kann es auf eine Vielzahl an Projekten aus den Bereichen Data Mining, Machine Learning, Data Science oder Analytics angewandt werden.

{\normalsize BEWERTUNG DES AZURE MACHINE LEARNING STUDIOS} \par
Bei Azure Machine Learnung Studio handelt es sich um ein Werkzeug, dass alle Grundfunktionen bereitstellt. Es ist selbst durch R- oder Python-Skripte erweiterbar und besitzt eine aktuelle und ausreichende Dokumentation, jedoch keine Versionsverwaltung. Das Studio wird für Vorstudien, Evaluationsprojekte und zum Lernen, nicht aber für produktive Systeme empfohlen.
