\documentclass[fontsize=12pt, paper=a4, headinclude, twoside=true, parskip=half+, pagesize=auto, numbers=noenddot, plainheadsepline, open=right, toc=listof, toc=bibliography, chapteratlists=0pt]{scrbook}


% Allgemeines
\usepackage[automark]{scrpage2} % Kopf- und Fußzeilen
\usepackage{amsmath,marvosym} % Mathesachen
\usepackage[T1]{fontenc} % Ligaturen, richtige Umlaute im PDF
\usepackage[utf8]{inputenc}% UTF8-Kodierung für Umlaute usw
\usepackage{perpage} %the perpage package
\MakePerPage{footnote} %the perpage package command
%\usepackage{caption}
\usepackage[Q=yes]{examplep}
\usepackage{lipsum}

% Schriften
\usepackage{setspace} % Zeilenabstand
\onehalfspacing % 1,5 Zeilen
\usepackage{lmodern}

% Schriften-Größen
\setkomafont{chapter}{\Huge\rmfamily} % Überschrift der Ebene
\setkomafont{section}{\Large\rmfamily}
\setkomafont{subsection}{\large\rmfamily}
\setkomafont{subsubsection}{\small\rmfamily}
\setkomafont{chapterentry}{\large\rmfamily} % Überschrift der Ebene in Inhaltsverzeichnis
\setkomafont{descriptionlabel}{\bfseries\rmfamily} % für description Umgebungen
\setkomafont{captionlabel}{\small\bfseries}
\setkomafont{caption}{\small}

% Sprache: Deutsch
\usepackage[ngerman]{babel} % Silbentrennung

% PDF
\usepackage[ngerman, breaklinks=true]{hyperref}
\usepackage[final]{microtype} % mikrotypographische Optimierungen
\usepackage{url}
\usepackage{pdflscape} % einzelne Seiten drehen können

% Tabellen
\usepackage{multirow} % Tabellen-Zellen über mehrere Zeilen
\usepackage{multicol} % mehre Spalten auf eine Seite
\usepackage{tabularx} % Für Tabellen mit vorgegeben Größen
\usepackage{longtable} % Tabellen über mehrere Seiten
\usepackage{array}
\usepackage{float}
\usepackage{booktabs}

% Diagramme
\usepackage{tikz}
\usepackage{pgfplotstable}
\usepackage{pgfplots}
\usetikzlibrary{trees}

%  Bibliographie
\usepackage{bibgerm} % Umlaute in BibTeX
%\usepackage{cite}
\usepackage[round]{natbib} 


% Bilder
\usepackage{graphicx} % Bilder
\usepackage{color} % Farben
\graphicspath{{images/}}
\DeclareGraphicsExtensions{.pdf,.png,.jpg} % bevorzuge pdf-Dateien
\usepackage{subfigure} % mehrere Abbildungen nebeneinander/übereinander
\newcommand{\subfigureautorefname}{\figurename} % um \autoref auch für subfigures benutzen
\usepackage[all]{hypcap} % Beim Klicken auf Links zum Bild und nicht zu Caption gehen
\usepackage{caption}

% Bildunterschrift
\usepackage{chngcntr}
%\counterwithout{figure}{chapter}
\setcapindent{0em} % kein Einrücken der Caption von Figures und Tabellen
\setcapwidth[c]{0.9\textwidth}
\setlength{\abovecaptionskip}{0.2cm} % Abstand der zwischen Bild- und Bildunterschrift

% Quellcode
\usepackage{listings} % für Formatierung in Quelltexten
\usepackage[T1]{fontenc}
\usepackage{lmodern}% better font than default
\usepackage{tgcursor}% tt font with bold/italic styles
\lstset{
    language=R,
    numbers=left,
    stepnumber=1,
    numberstyle=\scriptsize,
	numbersep=10pt,
    extendedchars=true,
	numberbychapter=true,
 	frame=l,
	captionpos=b,		
	xleftmargin=5pt,
	tabsize=2,
	breakautoindent  = true,
	breaklines       = true,
	breakatwhitespace = true,
    basicstyle=\scriptsize\ttfamily,
    stringstyle=\color{deepgreen},
    otherkeywords={0,1,2,3,4,5,6,7,8,9},
    morekeywords={TRUE,FALSE},
    deletekeywords={data,frame,length,as,character},
    keywordstyle=\color{blue},
    commentstyle=\color{deepgreen},
}

% Custom colors
\definecolor{deepblue}{rgb}{0,0,0.5}
\definecolor{deepred}{rgb}{0.6,0,0}
\definecolor{deepgreen}{rgb}{0,0.5,0}
\definecolor{grau}{rgb}{0.5,0.5,0.5}

% Default fixed font does not support bold face
\DeclareFixedFont{\ttb}{T1}{txtt}{bx}{n}{10} % for bold
\DeclareFixedFont{\ttm}{T1}{txtt}{m}{n}{10}  % for normal



	
% linksbündige Fußboten
\deffootnote{1.5em}{1em}{\makebox[1.5em][l]{\thefootnotemark}}


\typearea{14} % typearea berechnet einen sinnvollen Satzspiegel (das heißt die Seitenränder) siehe auch http://www.ctan.org/pkg/typearea. Diese Berechnung befindet sich am Schluss, damit die Einstellungen oben berücksichtigt werden
% für autoref von Gleichungen in itemize-Umgebungen
\makeatletter
\newcommand{\saved@equation}{}
\let\saved@equation\equation
\def\equation{\@hyper@itemfalse\saved@equation}
\makeatother 



% Eigene Befehle %%%%%%%%%%%%%%%%%%%%%%%%%%%%%%%%%%%%%%%%%%%%%%%%%5
% Matrix
\newcommand{\mat}[1]{
      {\textbf{#1}}
}

\newcommand{\todo}[1]{
      {\colorbox{red}{ TODO: #1 }}
}
\newcommand{\todotext}[1]{
      {\color{red} TODO: #1} \normalfont
}
\newcommand{\info}[1]{
      {\colorbox{blue}{ (INFO: #1)}}
}
% Hinweis auf Programme in Datei
\newcommand{\datei}[1]{
      {\ttfamily{#1}}
}
\newcommand{\code}[1]{
      {\ttfamily{#1}}
}
% bild mit defnierter Breite einfügen
\newcommand{\bildF}[3]{
  \begin{minipage}{\textwidth}
    \centering
      \vspace{1ex}
      \includegraphics[#2]{images/#1}
      \captionof{figure}{#3}\label{img.#1}      
      \vspace{1ex}
  \end{minipage}
}

% bild mit defnierter Breite einfügen
\newcommand{\bild}[3]{
  \begin{figure}[!h]
    \centering
      \vspace{1ex}
      \includegraphics[#2]{images/#1}
      \caption[#3]{#3}\label{img.#1}      
      \vspace{1ex}
  \end{figure}
}

\newcommand{\bildL}[2]{
  \begin{figure}[!hbt]
      \includegraphics[#2]{images/#1}
      \protect\label{img.#1} 
  \end{figure}
}

% thumbnail einfügen

\newcommand{\bildthumb}[2]{
 \begin{figure}[!hbt]
	\hfill\includegraphics[#2]{images/#1}
  \end{figure}	
}

%multicitep start
\usepackage{xparse}
\ExplSyntaxOn

\makeatletter
\NewDocumentCommand{\multicitep}{m}
 {
  \NAT@open
  \mjb_multicitep:n { #1 }
  \NAT@close
 }
\makeatother

\seq_new:N \l_mjb_multicite_in_seq
\seq_new:N \l_mjb_multicite_out_seq
\seq_new:N \l_mjb_cite_seq

\cs_new_protected:Npn \mjb_multicitep:n #1
 {
  \seq_set_split:Nnn \l_mjb_multicite_in_seq { ; } { #1 }
  \seq_clear:N \l_mjb_multicite_out_seq
  \seq_map_inline:Nn \l_mjb_multicite_in_seq
   {
    \mjb_cite_process:n { ##1 }
   }
  \seq_use:Nn \l_mjb_multicite_out_seq { ;~ }
 }

\cs_new_protected:Npn \mjb_cite_process:n #1
 {
  \seq_set_split:Nnn \l_mjb_cite_seq { , } { #1 }
  \int_compare:nTF { \seq_count:N \l_mjb_cite_seq == 1 }
   {
    \seq_put_right:Nn \l_mjb_multicite_out_seq
     { \citeauthor{#1},~\citeyear{#1} }
   }
   {
    \seq_put_right:Nx \l_mjb_multicite_out_seq
     {
      \exp_not:N \citeauthor{\seq_item:Nn \l_mjb_cite_seq { 1 }},~
      \exp_not:N \citeyear{\seq_item:Nn \l_mjb_cite_seq { 1 }},~
      \seq_item:Nn \l_mjb_cite_seq { 2 }
     }
   }
 }
\ExplSyntaxOff
%multicitep end

%german quotationmarks
\usepackage{csquotes}
\MakeOuterQuote{"}

%double line for tables
\usepackage{hhline}

%lange tabellen mit seitenumbruch
\usepackage{longtable}

%scientific math notation with e
\usepackage{siunitx}
\sisetup{output-exponent-marker=\ensuremath{\mathrm{e}}}

%für Kästen
\usepackage{framed}

%Offset fürs Binden
%\usepackage{geometry}
%\geometry{bindingoffset=1cm}

%glossar
\usepackage[nomain,acronym,xindy,toc,nonumberlist]{glossaries} % nomain, if you define glossaries in a file, and you use \include{INP-00-glossary}
\makeglossaries
\usepackage[xindy]{imakeidx}
\makeindex
\newglossaryentry{model}
{
    name=Model,
    description={von der englischen Bezeichung für ein mathematisches Modell}
}
\newglossaryentry{btc}
{
    name=BTC,
    description={Abkürzung für die Kryptowährung Bitcoin}
}
\newglossaryentry{eth}
{
    name=ETH,
    description={Abkürzung für die Kryptowährung Ethereum}
}
\newglossaryentry{crisp}
{
    name=CRISP-DM,
    description={CRoss Industrial Standard Process for Data Mining}
}
\newglossaryentry{kdd}
{
    name=KDD,
    description={Knowledge Discovery (in) Databases }
}
\newglossaryentry{semma}
{
    name=SEMMA,
    description={Sample, Explore, Modify, Model and Assess}
}
\newglossaryentry{eda}
{
    name=EDA,
    description={Explorative Data Analysis; Explorative Datenanalyse}
}
\newglossaryentry{saas}
{
    name=SaaS,
    description={Software as a Service}
}
\newglossaryentry{usd}
{
    name=USD,
    description={US-Dollar, United States Dollar}
}
\newglossaryentry{mrd}
{
    name=Mrd.,
    description={Milliarden, 1.000.000.000}
}
\newglossaryentry{soa}
{
    name=SOA,
    description={service oriented atchitekture, Service-orientierte Architektur}
}
\newglossaryentry{ml}
{
    name=ML,
    description={Machine Learning; Maschinelles lernen}
}
\newglossaryentry{rs}
{
    name=$ R^2 $,
    description={(adjusted) R squared; R-Quadrat; adjustiertes oder angepasstes Bestimmtheitsmaß; statistische Größe bei der Bewertung einer Regression}
}
\newglossaryentry{pca}
{
    name=PCA,
    description={Principal component analysis; Hauptkomponentenanalyse}
}
\newglossaryentry{obs}
{
    name=Observations,
    description={Beobachtungen; Reihen/Zeilen eines Datensatzes; records, rows}
}
\newglossaryentry{feat}
{
    name=Features,
    description={Spalten eines Datensatzes; attributes; Attribute; columns}
}
\newglossaryentry{train}
{
    name=train set,
    description={Teil eines Datensatzes, das zum Trainieren eines Models genutzt wird}
}
\newglossaryentry{test}
{
    name=test set,
    description={Teil eines Datensatzes, das zum Testen eines trainierten Models eingesetzt wird}
}
\newglossaryentry{nuggets}
{
    name=KDnuggets,
    description={Webseite zum Lernen von und für Diskussionen über Data Science}
}
\newglossaryentry{kaggle}
{
    name=kaggle,
    description={populäres Data Science Portal}
}
\newglossaryentry{ann}
{
    name=ANN,
    description={artifical neuronal network; Künstliches neuronales Netze oder künstliches neuronales Netzwerk}
}
\newglossaryentry{ether}
{
    name=Ether,
    description={Währungsbezeichnung im Ethereumnetzwerk}
}
\newglossaryentry{ide}
{
    name=IDE,
    description={Integrated Development Environment}
}
\newglossaryentry{r}
{
    name=R,
    description={Programmiersprache}
}
\newglossaryentry{py}
{
    name=Python,
    description={Programmiersprache}
}
\newglossaryentry{csv}
{
    name=CSV,
    description={comma seperated values}
}
\newglossaryentry{tsv}
{
    name=TSV,
    description={tab seperated values}
}
\newglossaryentry{cny}
{
    name=CNY,
    description={Renminbi Yuan; chinesische Währung}
}
\newglossaryentry{jpy}
{
    name=JPY,
    description={Yen; japanische Währung}
}
\newglossaryentry{eur}
{
    name=EUR,
    description={Euro; Währung der Europäischen Wirtschafts- und Währungsunion}
}
\newglossaryentry{gbp}
{
    name=GBP,
    description={Pfund Sterling; britisches Pfund; Währung des Vereinigten Königreichs}
}
\newglossaryentry{inr}
{
    name=INR,
    description={Indische Rupie; indische Währung}
}
\newglossaryentry{brl}
{
    name=BRL,
    description={Real; brasilianische Währung}
}
\newglossaryentry{stlfsi}
{
    name=STLFSI,
    description={St. Louis Fed Financial Stress Index}
}
\newglossaryentry{mice}
{
    name=MICE,
    description={Multivariate Imputation by Chained Equations}
}





