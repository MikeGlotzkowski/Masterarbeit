\chapter{Motivation}


\section{Bitcoin als Vorreiter der Krypotowährungen}
Geld online von einem Teilnehmer direkt zu einem Anderen senden, ohne dabei (Transaktions-)Gebühren für einen zwischengelagerten Finanz-Dienstleister zahlen zu müssen, ist der Gedanke hinter dem "Peer-To-Peer Electronic Cash System"\citep{nakamoto_bitcoin:_2008} Bitcoin. Obwohl es Teilnehmern ohne Aufwand möglich ist, dem Netzwerk beizutreten oder es wieder zu verlassen, ist es solange unangreifbar, solange ein Angreifer nicht dauerhaft über mehr Rechenkapazität verfügt, als das komplette restliche Netzwerk.\citep{nakamoto_bitcoin:_2008} Obgleich immer wieder Kritik an der tatsächlichen Anonymität im Bitcoinnetzwerk laut wird\citep{reid_analysis_2013,androulaki_evaluating_2013} werden beim Nutzen des Netzwerk keine persönlichen Informationen an ein Kreditinstitut (wie PayPal, Paydirekt, ApplePay oder Masterpass) weitergegeben.\newline
Neben Bitcoin hat sich deswegen eine Vielzahl an anderen, sogenannten Kryptowährungen entwickelt. Im Nachfolgenden wird dabei zwischen Bitcoin(\ref{subsec:Bitcoin}), Ethereum(\ref{subsec:Ethereum})\citep{wood_ethereum:_2014} und Altcoins (aus dem Englischen: alternative coin\citep{prableen_bajpai_altcoin_2014})(\ref{subsec:Altcoins}) unterschieden.




Bedeutung heutzutage; bisschen Geschichte; Ursprünge; -->Techniken dann später

Hier statista sachen, bekanntheit, volumen, umschlag volumen, andere abspaltungen --> ethereum (smart contracts etc. hier whitepaper aus zotero);
dezentrale systeme; sicherheit (satoshi bitcoin paper) --> sicher


\subsection{Bitcoin}\label{subsec:Bitcoin}

bekannteste cryptowährung; bekannt als vorreiter; medien etc.; auch hier statista


\subsection{Ethereum}\label{subsec:Ethereum}
aufstreben, smart contracts wie angesprochen; nicht nur copycat


\subsection{Altcoins (Litecoin, Dogecoin)}\label{subsec:Altcoins}
alternativen, warum hier nicht betrachtet; nur kopien...




\section{Machine Learning und Data Mining}
Was ist das; wozu nutzt man es; wo ist der Unterschied --> genaueres dann später

viele daten; auswertung; automatisierung; hardware anforderungen


\section{Cloud-Dienste und SaaS}
Wieso Cloud Dienste Nutzen; Warum nicht nur lokal? (brauch ich diesen Teil?); einfach vorstellen und dann azure ml studio