\chapter{Motivation}

\section{Thema der Arbeit}
In der vorliegenden Arbeit werden Einflussfaktoren auf den Kurs von ausgewählten Kryptowährungen gesucht und der Grad des Einflusses evaluiert. Dies geschieht mit dem Ziel herauszufinden, ob sich die Kursschwankungen der digitalen Währungen voraussagen lassen und wenn ja, in welchem Maße. Im nachfolgenden Kapitel wird auf die Gründe für die Analyse eingegangen. Das genaue Vorgehen und die Ziele werden wird in Abschnitt \ref{chapter:Vorgehen} erläutert.

\section{Bitcoin als Vorreiter der Krypotowährungen}
Geld online von einem Teilnehmer direkt zu einem Anderen senden, ohne dabei (Transaktions-)Gebühren für einen zwischengelagerten Finanz-Dienstleister zahlen zu müssen, ist der Gedanke hinter dem "Peer-To-Peer Electronic Cash System"\citep{nakamoto_bitcoin:_2008} Bitcoin. Obwohl es Teilnehmern ohne Aufwand möglich ist, dem Netzwerk beizutreten oder es wieder zu verlassen, ist es solange unangreifbar, solange ein Angreifer nicht dauerhaft über mehr Rechenkapazität verfügt, als das komplette restliche Netzwerk.\citep{nakamoto_bitcoin:_2008} Ob das Bitcoinnetzwerk wirklich absolute Anonymität gewährt, wird stark kritisiert.\citep{reid_analysis_2013,androulaki_evaluating_2013}. In der Tat werden beim Nutzen des Netzwerk jedoch keine persönlichen Informationen an ein Kreditinstitut (wie PayPal, Paydirekt, ApplePay oder Masterpass) weitergegeben. Diese Argumente (Kostenreduktion, Sicherheit und Anonymität) sorgen für Interesse an der digitalen Währung (auch hier gibt es Kritiker, die den Bitcoin als Investition und nicht als Währung bezeichnen)\citep{baur_bitcoin:_2015}. Nicht zu vernachlässigen ist an dieser Stelle auch das Interesse der Industrie an "Smart Contracts", die beispielsweise im Bereich des Internet of Things Anwedung finden.\citep{christidis_blockchains_2016}\newline
Neben Bitcoin hat sich deshalb zusätzlich eine Vielzahl an anderen sogenannten Kryptowährungen entwickelt. Die Währungen mit dem größten Marktvolumen sind  Bitcoin(\ref{subsec:Bitcoin}) und Ethereum(\ref{subsec:Ethereum})\citep{wood_ethereum:_2014}.\citep{brandt_infografik:_2017, coinmarketcap_ranking_2017} Daneben gibt es noch sogenannte Altcoins (aus dem Englischen: alternative coin\citep{prableen_bajpai_altcoin_2014})(\ref{subsec:Altcoins}). Mittlerweile umfassen diese 664 Bitcoin-Alternativen.\citep{coindesk_anzahl_2017}. Obgleich die tatsächliche Nutzung der Krypowährungen sehr gering ist (1\% der Befragten in Deutschland\citep{tsys_kennen_2016}), steigt das Interesse an Kryptowährungen\citep{wikitrends_compare_2017,googletrends_googletrends_2017}.\newline
\todo{irgendwas zu Technik später oder so?}


\todo{brauche ich diese Punkte hier wirklich?}
\subsection{Bitcoin}\label{subsec:Bitcoin}

bekannteste cryptowährung; bekannt als vorreiter; medien etc.; auch hier statista


\subsection{Ethereum}\label{subsec:Ethereum}
aufstreben, smart contracts wie angesprochen; nicht nur copycat


\subsection{Altcoins (Litecoin, Dogecoin)}\label{subsec:Altcoins}
alternativen, warum hier nicht betrachtet; nur kopien...




\section{Machine Learning, Data Mining, Data Analysis und Data Science}
Die Begriffe Machine Learning, Data Mining, Data Analysis und Data Science stellen zwar verwandte Begriffe dar, sind jedoch alles eigene Themenfelder mit Berührungspunkten. \newline
Machine Learning gehört zur Familie der Künstlichen Intelligenz in der Informatik und Mathematik.\multicitep{kim_matlab_2017, S.~2; swamynathan_mastering_2017, S.~54}. Es kann als "Sammlung Algorithmen und Techniken" verstanden werden, die "genutzt werden, um Computersysteme zu erstellen, die aus Daten lernen, um Vorhersagen zu machen" \citep[S.~53]{kim_matlab_2017}


Was ist das; wozu nutzt man es; wo ist der Unterschied --> genaueres dann später

viele daten; auswertung; automatisierung; hardware anforderungen


\section{Cloud-Dienste und SaaS}
Wieso Cloud Dienste Nutzen; Warum nicht nur lokal? (brauch ich diesen Teil?); einfach vorstellen und dann azure ml studio