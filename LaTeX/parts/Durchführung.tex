\chapter{Durchführung der Analyse}
Nachdem in den vorhergegangen Abschnitten alle Aspekte des Themas "Kursanalyse von Kryptowährungen mit Azure Machine Learning" betrachtet wurden, widmet sich dieser Abschnitt der praktischen Umsetzung. Jeder Themenblock hat seinen Teil zum Erstellen eines Kontextes beigetragen, in dem die Analyse durchgeführt werden kann (siehe Tabelle \ref{tab:themeblocks}).
\begin{table}[H]
\begin{tabular}{|p{3,5cm}|p{6cm}|p{7cm}|}
\hline
\textbf{Themenblock} & \textbf{Inhalt} & \textbf{Ziel}\\ 
\hhline{===}
Data Mining Frameworks (\ref{sec:DataMiningFrameworks}) & Beschreibung der bekanntesten Frameworks des Data Mining Prozesses und Auswahl eines Frameworks für die vorliegende Arbeit & Detailliertes Beschreiben des nachfolgenden Prozessmodells\\
\hline
Machine Learning (\ref{sec:MachineLearning}) & Vorstellung einer Möglichkeit zur Einordnung von Machine Learning Typen und Algorithmen & Verständnisaufbau für die Analyse in diesem Teil der Arbeit\\
\hline
Kryptowährungen (\ref{sec:cryptocurrency2}) & Hintergrundwissen zu Kryptowährungen & Für eine Analyse ist Hintergrundwissen wichtig. Dieses sogenannte "Domain"-Wissen ist essentieller Bestandteil einer Analyse mit CRISP-DM\\
\hline
Microsoft Azure Machine Learning Studio (\ref{sec:msmls}) & Allgemeine Beschreibunng und Aufbau des Werkzeugs & Das Studio soll als Werkzeug zur Analyse eingesetzt werden.\\
\hline
\end{tabular}
\caption{Behandelte theoretische Abschnitte im Kontext der Arbeit}
\label{tab:themeblocks}
\end{table}
Wie in Punkt \ref{sec:crispdmdec} angesprochen, wird als Hilfe für das Prozessmodell CRISP-DM der zugehörige User Guide \citep[S.~30-56]{chapman_crisp-dm_2000} herangezogen. Die, im Guide genannten, Outputs jedes Prozessschritts werden nachfolgend speziell hervorgehoben.
 
\section{Business Understanding}
\subsection{Determine the Business Objectives}
Das CRISP-DM Prozessmodell ist sehr generisch gehalten. Dies ist beispielsweise daran zu erkennen, dass das Modell in vier übereinander liegende Abstraktionsschichten gegliedert ist.\citep[S.~6]{chapman_crisp-dm_2000} Dies dient der Anpassungsfähigkeit an viele heterogene Projekte. Diese Anpassungsfähigkeit wird gleich im ersten Schritt genutzt.\newline
In  gewöhnlichen Industrie- oder Forschungsprojekten ist es wichtig, die Stakeholder (vor allem Geldgeber) und den Reifegrad und die Akzeptanz des Data Mining im Projektumfeld zu analysieren. Dies Rückt im vorliegenden Fall in den Hintergrund. Die Anderen Outputs sind jedoch ebenso wichtig.

\begin{longtable}[H]{|p{4,5cm}|p{12cm}|}
\hline
\textbf{Output} & \textbf{Beschreibung} \\ 
\hhline{==}
Background & Die Analyse wird im Rahmen einer Masterarbeit durchgeführt. Nur eine Person ist daran beteiligt. \\
\hline
Business objectives & Die Untersuchung hat zwei Hauptziele. Das eine ist die Analyse der Cryptowährungen an sich. Es soll herausgefunden werden, ob Kursschwankungen mit Hilfe der Isolation von Einflussfaktoren und den Mitteln des Machine Learning vorausgesagt werden können oder anderweitige Auffälligkeiten zu beobachten sind. Das andere Ziel ist die Einarbeitung in das Werkzeug Azure Machine Learning. \\
\hline
Business success criteria & Die Erkenntnis, dass eine Vorhersage nicht möglich ist, oder dass wichtige Einflüsse nicht gefunden wurden, ist durchaus möglich und bedeutet keinesfalls ein Scheitern des Projekts. Hinsichtlich des Werkzeugs Azure ML, ist es beispielsweise interessant, welchen Restriktionen das Tool unterlegen ist. Das Betrifft sowohl Funktionen, die (noch) nicht vorhanden sind, oder technische Limitationen, wie Geschwindigkeit, Volumenbegrenzungen etc..\\
\hline
\caption{Output des Schrittes "Determine the Business Objectives"}
\end{longtable}

\subsection{Assess the Situation}
Dieser Teil befasst sich vor allem damit, welche Ressourcen zur Verfügung stehen (Hardware, Software, personell) und welche sonstigen Bedingungen erfüllt sein müssen oder das Projekt begrenzen. Dazu zählt auch das Finden von Daten, die für die Modellierung genutzt werden können. Anzumerken ist hierbei, dass es hier noch nicht um das tatsächliche Laden der Daten im Sinne von Dateien geht, sondern um das Finden von potentiellen Quellen für diese Daten. Zusätzlich sollen noch eine Risiko- und eine Kosten-Nutzen-Analyse durchgeführt werden. Das Hauptaugenmerk liegt jedoch auf der Erschließung der Daten. \newline







Kosten-Nutzen-Analyse

Risiko-Analyse

Ziel: Frage detaillieren; welche Ressourcen zur verfügbar sind; grenzen der analyse; annahmen
--> kann sein, dass es mit anderen werkzeugen (tenser flow etc.) einfacher/besser geht, aber das nicht ziel
--> ressourcen knapp; eine person, umfang masterarbeit
--> Welche Daten kommen in Frage?

--------
aus paper und mehr suchen

welchen einluss hier; im nächsten teil dann:
wie kann man das repräsentieren, welche daten gibt es da und kann man das abbilden?

beispiele:
regierungen und regionen (usa, china, EU) --> Gesetze

bitcoin-eigene dinge (volumen, umschlag, miner? etc.)

öffentlichkeit (twitter, zeitungen, blogs, domains im web)

natürliche Ressourcen (Öl, Gold, Silber, Diamanten w/e)

Financial Stress Index (FSI)


HIER PAPER NOCHMAL:
* Economic Drivers
* Transaction Drivers
* Technical Drivers
* Interest
* Safe Haven
* Influence of China
------

\begin{longtable}[H]{|p{6,5cm}|p{10cm}|}
\hline
\textbf{Output} & \textbf{Beschreibung} \\ 
\hhline{==}
Inventory of resources & Personal
\begin{itemize}
\item 1 Person mit Zugang zu den Recherche Ressourcen der Hochschule München (OPAC, DBIS, ZDB etc.\citep{noauthor_hochschule_2017})
\end{itemize}
Hardware
\begin{itemize}
\item 1 PC (CPU: AMD Ryzen 5 1600 Sechskern; RAM: 8GB; GPU: NVIDIA GeForce GTX 1060 (6GB VRAM); Windows 10 Education Build 15063.674)
\end{itemize}
Software
\begin{itemize}
\item 1 "Free"-Account Microsoft Azure Machine Learning Studio mit Workspace in "South Central US"
\item Version des Juypter Notebooks 5.1.0
\item Auf dem PC: R Version 3.4.1
\item Auf dem PC: RStudio Version 1.0.153
\end{itemize}
Daten
\begin{itemize}
\item DS 1
\item DS 2
\end{itemize}
\\
\hline
Requirements, assumptions, and constraints &  Zu Bedenken ist, dass bei einer kostenlosen Subscription im Azure ML Studio nur 10GB Storage verfügbar sind. Zusätzlich müssen Daten, die analysiert werden sollen in das Tool geladen werden. Bei einem Upload vom PC, wird das durch die Upload-Bandbreite limitiert. Eventuell müssen Daten im Projektverlauf auch öfter Hochgeladen werden, was zu Verzögerungen führen könnte. Außerdem lässt sich in der freien Version nur jeweils ein Experiment gleichzeitig ausführen. Das parallele Trainieren von Modellen ist somit nicht möglich.\newline 
Obwohl zur Hilfe neben der Web-IDE auch RStudio genutzt werden kann, soll die Untersuchung hauptsächlich mit den Azure ML Studio Bordmitteln durchgeführt werden.\newline
Schließlich sollen nur solche Daten genutzt werden, die entweder frei verfügbar sind oder mit dem Hochschulzugang zu beschaffen sind. Das schließt präparierte Datensätze von Bezahlseiten aus.\\
\hline
Risks and contingencies & yyy \\
\hline
Terminology & yyy \\
\hline
Costs and benefits & yyy \\
\hline
\caption{Output des Schrittes "Assess the Situation"}
\end{longtable}


\subsection{Determine the Data Mining Goals}
Ziel: technische Beschreibung der Ziele
--> Find any Patterns, correlations, trends etc. in crypto data (1: de-/increase; 2: dollar exchange rate)

\subsection{Produce a Project Plan}
Ziel: Projektplan erstellen
--> Schritte, die durchgeführt werden (Gliederung hier ca)
--> Timeline (wie verältnis; evtl. auslassen?)
--> Risiken (gar nichts finden; Daten Fehlen; techn. Probleme mit Azure...)
--> Werkzeuge und Techniken (R, RStudio, Jupyter Notebooks, Azure ML, Excel)

\section{Data Understanding}
\subsection{Collect the Initial Data}
Ziel: daten beschaffen
--> oben, welche Sachen hätte ich gerne; hier dann richtige Daten und schon in azure laden!

\subsection{Describe the Data}
Ziel: Format, größe, Anzahl der Beobachtungen; Beschaffenheit
-->
--> sind die Daten okay? 

\subsection{Explore the Data}
Ziel: Einfache Analyse der Zielvariablen 
--> (steigend, fallend, preis?)

\subsection{Verify Data Quality}
Ziel: Qualität sichern; common sense
--> alle preise in usd; zeit in ttmmyyyy oder so
--> daten vollständig für die sachen?
--> kommas und punkte richtig?

\section{Data Preperation}
\subsection{Select Data}
Ziel: Welche Daten sind wichtig für die Analyse?
--> Vlt Daten wochenweise aggregieren
--> 

\subsection{Clean Data}
Ziel: Sachen aus "Verfiy Data Quality" aufgreifen;
--> Lücken füllen, Preise, Datum

\subsection{Construct Data}
Ziel: Daten so umbauen, dass sie in algorithmus passen
--> neue Zeilen hinzufügen; Verdoppeln w/e

\subsection{Integrate Data}
Ziel: Daten zusammenführen
--> die fertigen Datensätze, die ich gefunden habe in die BTC-, ETH-Kursdaten überführen

\subsection{Format Data}
evtl. auspassen

\section{Modeling}
\subsection{Select the Modeling Technique}
Ziel: Auswahl der Algorithmen
--> Supervised (Continous value --> wert; two-class (hoch/runter))

\subsection{Generate Test Design}
Ziel: Training und error rates betrachten
--> azure split data experiment item!

\subsection{Build the Model}
Ziel: Models bauen
--> "azure ml run"

\subsection{Assess the Model}


\section{Evaluation}
\subsection{Evaluate Results}
\subsection{Review Process}
\subsection{Determine Next Steps}


\section{Deployment}
\subsection{Plan Deployment}
\subsection{Plan Monitoring and Maintenance}
\subsection{Produce Final Report}
\subsection{Review Project}


