\chapter{Durchführung der Analyse}
Nachdem in den vorhergegangen Abschnitten alle Aspekte des Themas "Kursanalyse von Kryptowährungen mit Azure Machine Learning" betrachtet wurden, widmet sich dieser Abschnitt der praktischen Umsetzung. Jeder Themenblock hat seinen Teil zum Erstellen eines Kontextes beigetragen, in dem die Analyse durchgeführt werden kann (siehe Tabelle \ref{tab:themeblocks}).
\begin{table}[H]
\begin{tabular}{|p{3,5cm}|p{6cm}|p{7cm}|}
\hline
\textbf{Themenblock} & \textbf{Inhalt} & \textbf{Ziel}\\ 
\hhline{===}
Data Mining Frameworks (\ref{sec:DataMiningFrameworks}) & Beschreibung der bekanntesten Frameworks des Data Mining Prozesses und Auswahl eines Frameworks für die vorliegende Arbeit & Detailliertes Beschreiben des nachfolgenden Prozessmodells\\
\hline
Machine Learning (\ref{sec:MachineLearning}) & Vorstellung einer Möglichkeit zur Einordnung von Machine Learning Typen und Algorithmen & Verständnisaufbau für die Analyse in diesem Teil der Arbeit\\
\hline
Kryptowährungen (\ref{sec:cryptocurrency2}) & Hintergrundwissen zu Kryptowährungen & Für eine Analyse ist Hintergrundwissen wichtig. Dieses sogenannte "Domain"-Wissen ist essentieller Bestandteil einer Analyse mit CRISP-DM\\
\hline
Microsoft Azure Machine Learning Studio (\ref{sec:msmls}) & Allgemeine Beschreibunng und Aufbau des Werkzeugs & Das Studio soll als Werkzeug zur Analyse eingesetzt werden.\\
\hline
\end{tabular}
\caption{Behandelte theoretische Abschnitte im Kontext der Arbeit}
\label{tab:themeblocks}
\end{table}
Wie in Punkt \ref{sec:crispdmdec} angesprochen, wird als Hilfe für das Prozessmodell CRISP-DM der zugehörige User Guide \citep[S.~30-56]{chapman_crisp-dm_2000} herangezogen. Die, im Guide genannten, Outputs jedes Prozessschritts werden nachfolgend speziell hervorgehoben. Das CRISP-DM Prozessmodell ist sehr generisch gehalten. Dies ist beispielsweise daran zu erkennen, dass das Modell in vier übereinander liegende Abstraktionsschichten gegliedert ist.\citep[S.~6]{chapman_crisp-dm_2000} Dies dient der Anpassungsfähigkeit an viele heterogene Projekte. Diese Anpassungsfähigkeit wird gleich im ersten Schritt genutzt.\newline
 
\section{Business Understanding}
\subsection{Determine the Business Objectives}
In  gewöhnlichen Industrie- oder Forschungsprojekten ist es wichtig, die Stakeholder (vor allem Geldgeber) und den Reifegrad und die Akzeptanz des Data Mining im Projektumfeld zu analysieren. Dies rückt im vorliegenden Fall in den Hintergrund. Die Anderen Outputs sind jedoch ebenso wichtig.

\begin{longtable}[H]{|p{4,5cm}|p{12cm}|}
\hline
\textbf{Output} & \textbf{Beschreibung} \\ 
\hhline{==}
Background & Die Analyse wird im Rahmen einer Masterarbeit durchgeführt. Nur eine Person ist daran beteiligt. \\
\hline
Business objectives & Die Untersuchung hat zwei Hauptziele. Das eine ist die Analyse der Cryptowährungen an sich. Es soll herausgefunden werden, ob Kursschwankungen mit Hilfe der Isolation von Einflussfaktoren und den Mitteln des Machine Learning vorausgesagt werden können oder anderweitige Auffälligkeiten zu beobachten sind. Das andere Ziel ist die Einarbeitung in das Werkzeug Azure Machine Learning. \\
\hline
Business success criteria & Die Erkenntnis, dass eine Vorhersage nicht möglich ist, oder dass wichtige Einflüsse nicht gefunden wurden, ist durchaus möglich und bedeutet keinesfalls ein Scheitern des Projekts. Hinsichtlich des Werkzeugs Azure ML, ist es beispielsweise interessant, welchen Restriktionen das Tool unterlegen ist. Das Betrifft sowohl Funktionen, die (noch) nicht vorhanden sind, oder technische Limitationen, wie Geschwindigkeit, Volumenbegrenzungen etc..\\
\hline
\caption{Output des Schrittes "Determine the Business Objectives"}
\end{longtable}

\subsection{Assess the Situation}
Dieser Teil befasst sich vor allem damit, welche Ressourcen zur Verfügung stehen (Hardware, Software, personell) und welche sonstigen Bedingungen erfüllt sein müssen oder das Projekt begrenzen. Dazu zählt auch das Finden von Daten, die für die Modellierung genutzt werden können. Anzumerken ist hierbei, dass es noch nicht um das tatsächliche Laden der Daten im Sinne von Dateien geht, sondern um das Finden von potentiellen Quellen für Daten. Zusätzlich sollen noch eine Risiko- und eine Kosten-Nutzen-Analyse durchgeführt werden. Das Hauptaugenmerk liegt jedoch auf der Erschließung der Daten. 
\newline
An Stelle einer vollständigen Risikoanalyse (Kontext herstellen, Risiken identifizieren, analysieren, evaluieren, managen\citep[S.~43]{sowa_management_2017}), tritt eine Aufzählung der 3 Hauptrisiken. Dies geschieht einerseits aus Gründen der Verhältnismäßigkeit, andererseits liegt der Fokus der Arbeit auf einem anderen Thema. Ähnlich verhält es sich mit der Kosten-Nutzen-Analyse. Sie wird zur Bewertung der Wirtschaftlichkeit herangezogen, was in diesem Kontext nicht relevant ist. Deswegen wird auf sie vollständig verzichtet.
\newline
Die schwerste Aufgabe dieses Teils ist das Finden von Daten, die Einfluss auf den Kurs von Kryptowährungen haben (könnten). Forschungen in diesem Bereich identifizierten einerseits öffentliches Interesse (soziale Medien, Google Suchanfragen, etc.)\multicitep{kristoufek_bitcoin_2013; garcia_digital_2014} und andererseits auch wirtschaftliche Faktoren ("standard economic theory", also Angebot und Nachfrage, Investoren)\citep{kristoufek_what_2015} als Haupteinflussfaktoren. Anhand den Aussagen dieser Paper und zusätzlichen Überlegungen ergeben sich folgende Faktoren, die bei der Analyse bedacht werden (Tabelle \ref{tab:dataToAnalyse}).\newline Darüber hinaus wird die Quelle aufgeführt, über die die Daten bezogen werden. Bei monetären Strömen oder Kursen, wird immer der Kurs in USD herangezogen. 
\newline
Damit eine Analyse der Kryptowährungen möglich ist, müssen zu diesen historische Kursdaten beschafft werden. Dieser, auf den ersten Blick simpel wirkende Schritt, ist in Wirklichkeit \todo{formulierung} nicht trivial. Kryptowährungen werden an dutzenden Portalen gleichzeitig gehandelt. Dabei erscheinen genauso schnell neue Börsen, wie alte verschwinden. Auch das Handelsvolumen und die Handelswährung unterscheidet sich. Hinzu kommt, dass an Bitcoin- oder Ethereum-Handelsportalen meist 24 Stunden täglich gehandelt werden kann. Aus diesem Grund sind solche Datenquellen ausgewählt worden, die entweder ihre Daten direkt von der Blockchain erhalten oder einen gewichteten Mittelwert über die größten Handelsplatformen berechnen.
\newline
Die nächste Schwierigkeit ist die Auswahl der Aktienindizes, die für die Analyse herangezogen werden. Es existiert keine allgemein gültige oder anerkannte Liste mit 'den wichtigsten Aktienindizes'. Aus diesem Grund wurden die Indices ausgewählt, die die Website investing.com als "major world indices" deklariert.\citep{fusion_media_limited_major_2017} Es liegt dabei eine große Überschneidung mir anderen Stock Market-Seiten vor.\multicitep{liveindex.org_live_2017; yahoo_finance_major_2017; yahoo_finance_major_2017-1} Zusätzlich zu den Aktienindices wird ein Financial Stress Index (FSI) herangezogen. Ein solcher "index misst die aktuelle Belastung in einem finanzwirtschaftlichen System"\citep[S.~1;eigene Übersetzung]{vermeulen_financial_2014} In diesem Fall ist es der St. Louis Fed Financial Stress Index, der 18 Einzelfaktoren aus drei Kategorien bündelt.\citep{federal_reserve_bank_of_st._louis_st._2017}
\newline
Ferner werden die Währungen der acht größten Volkswirtschaften\citep{the_international_monetary_fund_world_2017} und die Kurse für Gold, Silber und Rohöl mit einbezogen. Es werden die Wechselkurse zum Dollar betrachtet, sprich Fremdwährung/USD. Bei den Ölkursen wird Brent, das wichtigste Rohöl für den europäischen Markt \citep{noauthor_brent_2016}, und West Texas Intermediate (WTI), das Pondon für den US-Markt, betrachtet\citep{noauthor_west_2017}.

\begin{longtable}[H]{|p{4cm}|p{6,25cm}|p{6,25cm}|}
\multicolumn{3}{c}{\textit{Cryptowährungs-eigene Faktoren}}\\ \hline
\textbf{Daten} & \textbf{für BTC} & \textbf{für ETH} \\
\hhline{===}
Handelsvolumen & ja, von https://bitcoincharts.com/ &  ja, von https://coinmarketcap.com/ \\ \hline
Coin Volumen (Gesamtanzhal der vorhandenen Bitcoins/des Ethers) & ja, von https://blockchain.info/ & ja, von https://etherscan.io/ \\ \hline
Mining-Schwierigkeit \todo{Erklärung} & ja, von https://data.bitcoinity.org/ & ja, von https://etherscan.io/ \\ \hline
Anzahl der Transaktionen & ja, von https://blockchain.info/ & ja, von https://etherscan.io/ \\ \hline
Hashrate & ja, von https://www.kaggle.com/ & ja, von https://www.kaggle.com/ \\ \hline
Marktkapitalisierung & ja, von https://www.kaggle.com/ & ja, von https://www.kaggle.com/ \\ \hline
\multicolumn{3}{c}{\textit{Öffentliches Interesse}}\\ \hline
\textbf{Daten} & \textbf{für BTC} & \textbf{für ETH} \\
\hhline{===}
Google Websuchen & ja, von https://trends.google.de/trends/ & ja, von https://trends.google.de/trends/ \\ \hline
Google News-Suchen & ja, von https://trends.google.de/trends/ & ja, von https://trends.google.de/trends/ \\ \hline
Wikipedia Seitenaufrufe & ja, von https://wikimedia.org/api/rest\_v1/ & ja, von https://wikimedia.org/api/rest\_v1/ \\ \hline
Tweets (Twitter Nachrichten) & \multicolumn{2}{c}{nein, nicht kostenlos verfügbar} \\ \hline
Zeitungsartikel/-Überschriften & \multicolumn{2}{c}{ja, von https://www.kaggle.com/sunnysai12345/news-summary}\\ \hline
Blogartikel & \todo{???} & \todo{???} \\ \hline
(Web-)Domains & \todo{???} & \todo{???} \\ \hline
\multicolumn{3}{c}{\textit{(Aktien)indizes}}\\ \hline
\textbf{Daten} & \textbf{für BTC} & \textbf{für ETH} \\
\hhline{===}
Dow 30	& \multicolumn{2}{c}{ja, von https://www.investing.com/indices/}\\ \hline
S\&P 500	& \multicolumn{2}{c}{ja, von https://www.investing.com/indices/}\\ \hline
Nasdaq	 & \multicolumn{2}{c}{ja, von https://www.investing.com/indices/}\\ \hline
SmallCap & \multicolumn{2}{c}{ja, von https://www.investing.com/indices/}\\ \hline
S\&P 500 VIX& \multicolumn{2}{c}{ja, von https://www.investing.com/indices/}\\ \hline
S\&P/TSX	& \multicolumn{2}{c}{ja, von https://www.investing.com/indices/}\\ \hline
TR Canada  & \multicolumn{2}{c}{ja, von https://www.investing.com/indices/}\\ \hline
Bovespa	& \multicolumn{2}{c}{ja, von https://www.investing.com/indices/}\\ \hline
IPC & \multicolumn{2}{c}{ja, von https://www.investing.com/indices/}\\ \hline
DAX	& \multicolumn{2}{c}{ja, von https://www.investing.com/indices/}\\ \hline
FTSE 100		& \multicolumn{2}{c}{ja, von https://www.investing.com/indices/}\\ \hline
CAC 40	 	& \multicolumn{2}{c}{ja, von https://www.investing.com/indices/}\\ \hline
Euro Stoxx 	& \multicolumn{2}{c}{ja, von https://www.investing.com/indices/}\\ \hline
AEX		& \multicolumn{2}{c}{ja, von https://www.investing.com/indices/}\\ \hline
IBEX 	& \multicolumn{2}{c}{ja, von https://www.investing.com/indices/}\\ \hline
FTSE MIB	& \multicolumn{2}{c}{ja, von https://www.investing.com/indices/}\\ \hline	
SMI		& \multicolumn{2}{c}{ja, von https://www.investing.com/indices/}\\ \hline
PSI 	& \multicolumn{2}{c}{ja, von https://www.investing.com/indices/}\\ \hline
BEL 	& \multicolumn{2}{c}{ja, von https://www.investing.com/indices/}\\ \hline	 
ATX		& \multicolumn{2}{c}{ja, von https://www.investing.com/indices/}\\ \hline
OMXS30		& \multicolumn{2}{c}{ja, von https://www.investing.com/indices/}\\ \hline
OMXC20	& \multicolumn{2}{c}{ja, von https://www.investing.com/indices/}\\ \hline	
MICEX		& \multicolumn{2}{c}{ja, von https://www.investing.com/indices/}\\ \hline
RTSI	& \multicolumn{2}{c}{ja, von https://www.investing.com/indices/}\\ \hline	
WIG20		& \multicolumn{2}{c}{ja, von https://www.investing.com/indices/}\\ \hline
Budapest SE		& \multicolumn{2}{c}{ja, von https://www.investing.com/indices/}\\ \hline
BIST 100	& \multicolumn{2}{c}{ja, von https://www.investing.com/indices/}\\ \hline	
TA 35		& \multicolumn{2}{c}{ja, von https://www.investing.com/indices/}\\ \hline	 
Tadawul All Share	& \multicolumn{2}{c}{ja, von https://www.investing.com/indices/}\\ \hline
Nikkei 225		& \multicolumn{2}{c}{ja, von https://www.investing.com/indices/}\\ \hline
S\&P/ASX 200		& \multicolumn{2}{c}{ja, von https://www.investing.com/indices/}\\ \hline
DJ New Zealand		& \multicolumn{2}{c}{ja, von https://www.investing.com/indices/}\\ \hline
Shanghai	& \multicolumn{2}{c}{ja, von https://www.investing.com/indices/}\\ \hline
SZSE Component		& \multicolumn{2}{c}{ja, von https://www.investing.com/indices/}\\ \hline
China A50		& \multicolumn{2}{c}{ja, von https://www.investing.com/indices/}\\ \hline
DJ Shanghai	& \multicolumn{2}{c}{ja, von https://www.investing.com/indices/}\\ \hline
Hang Seng		& \multicolumn{2}{c}{ja, von https://www.investing.com/indices/}\\ \hline
Taiwan Weighted		& \multicolumn{2}{c}{ja, von https://www.investing.com/indices/}\\ \hline
SET		& \multicolumn{2}{c}{ja, von https://www.investing.com/indices/}\\ \hline
KOSPI	& \multicolumn{2}{c}{ja, von https://www.investing.com/indices/}\\ \hline
IDX Composite	& \multicolumn{2}{c}{ja, von https://www.investing.com/indices/}\\ \hline
Nifty 	& \multicolumn{2}{c}{ja, von https://www.investing.com/indices/}\\ \hline
BSE Sensex	& \multicolumn{2}{c}{ja, von https://www.investing.com/indices/}\\ \hline
PSEi Composite	& \multicolumn{2}{c}{ja, von https://www.investing.com/indices/}\\ \hline
STI Index	& \multicolumn{2}{c}{ja, von https://www.investing.com/indices/}\\ \hline
Karachi	& \multicolumn{2}{c}{ja, von https://www.investing.com/indices/}\\ \hline
HNX 30	& \multicolumn{2}{c}{ja, von https://www.investing.com/indices/}\\ \hline
CSE All-Share	& \multicolumn{2}{c}{ja, von https://www.investing.com/indices/}\\ \hline


St. Louis Fed Financial Stress Index (STLFSI) & \multicolumn{2}{c}{ja, von https://fred.stlouisfed.org/series/STLFSI}\\ \hline
\multicolumn{3}{c}{\textit{Währungen der größten Volkswirtschaften (nach BIP)}}\\ \hline
\textbf{Daten} & \textbf{für BTC} & \textbf{für ETH} \\
\hhline{===}
China (CNY) & \multicolumn{2}{c}{ja, von https://www.investing.com/currencies/single-currency-crosses}\\ \hline
Japan (JPY)& \multicolumn{2}{c}{ja, von https://www.investing.com/currencies/single-currency-crosses}\\ \hline
Deutschland (EUR) & \multicolumn{2}{c}{ja, von https://www.investing.com/currencies/single-currency-crosses}\\ \hline
Großbritannien  (GBP) & \multicolumn{2}{c}{ja, von https://www.investing.com/currencies/single-currency-crosses}\\ \hline
Frankreich (EUR) & \multicolumn{2}{c}{ja, von https://www.investing.com/currencies/single-currency-crosses}\\ \hline
Indien (INR) & \multicolumn{2}{c}{ja, von https://www.investing.com/currencies/single-currency-crosses}\\ \hline
Brasilien (BRL) & \multicolumn{2}{c}{ja, von https://www.investing.com/currencies/single-currency-crosses}\\ \hline
\multicolumn{3}{c}{\textit{natürliche Ressourcen}}\\ \hline
\textbf{Daten} & \textbf{für BTC} & \textbf{für ETH} \\
\hhline{===}
Goldpreis & \multicolumn{2}{c}{ja, von https://www.investing.com/commodities/}\\ \hline
Silberpreis & \multicolumn{2}{c}{ja, von https://www.investing.com/commodities/}\\ \hline
Brent (Rohöl Europa) & \multicolumn{2}{c}{ja, von https://www.investing.com/commodities/}\\ \hline
WTI (Rohöl USA) & \multicolumn{2}{c}{ja, von https://www.investing.com/commodities/}\\ \hline
\caption{Mögliche Einflussfaktoren auf den Kurs von Kryptowährungen}
\label{tab:dataToAnalyse}
\end{longtable}

\begin{longtable}[H]{|p{6,5cm}|p{10cm}|}
\hline
\textbf{Output} & \textbf{Beschreibung} \\ 
\hhline{==}
Inventory of resources & Personal
\begin{itemize}
\item 1 Person mit Zugang zu den Recherche Ressourcen der Hochschule München (OPAC, DBIS, ZDB etc.\citep{noauthor_hochschule_2017})
\end{itemize}
Hardware
\begin{itemize}
\item 1 PC (CPU: AMD Ryzen 5 1600 Sechskern; RAM: 8GB; GPU: NVIDIA GeForce GTX 1060 (6GB VRAM); Windows 10 Education Build 15063.674)
\end{itemize}
Software
\begin{itemize}
\item 1 "Free"-Account Microsoft Azure Machine Learning Studio mit Workspace in "South Central US"
\item Version des Juypter Notebooks 5.1.0
\item Auf dem PC: R Version 3.4.1
\item Auf dem PC: RStudio Version 1.0.153
\item Excel 2016 (Microsoft Office 365 ProPlus) Version 1710
\item Notepad++ Version 7.5.1
\end{itemize}
Daten
\begin{itemize}
\item in Tabelle \ref{tab:dataToAnalyse} genannte Daten
\item Kurse BTC/USD und ETH/USD
\item Zusätzliche Eigenschaften von Bitcoin und Ethereum. \citep{srk_cryptocurrency_2017} stellt unter der 'CC0: Public Domain'-Lizenz einen Datensatz zur Verfügung, der besondere Eigenschaften der Währungen enthält. Beispiele für Bitcoin sind die "Anzahl der einzigartigen Adressen" in der "Bitcoin Blockchain"; oder die "Anzahl der uncles pro Tag"\citep[eigene Übersetzung]{srk_cryptocurrency_2017} für Ethereum.
\end{itemize}
\\
\hline
Requirements, assumptions, and constraints & Zu Bedenken ist, dass bei einer kostenlosen Subscription im Azure ML Studio nur 10GB Storage verfügbar sind. Zusätzlich müssen Daten, die analysiert werden sollen, in das Tool geladen werden. Bei einem Upload vom PC wird das durch die Upload-Bandbreite limitiert. Eventuell müssen Daten im Projektverlauf auch öfter Hochgeladen werden, was zu Verzögerungen führen könnte. Außerdem lässt sich in der freien Version nur jeweils ein Experiment gleichzeitig ausführen. Das parallele Trainieren von Modellen ist somit nicht möglich.\newline 
Obwohl zur Hilfe neben der Web-IDE auch RStudio genutzt werden kann, soll die Untersuchung hauptsächlich mit den Azure ML Studio Bordmitteln durchgeführt werden. \newline
Schließlich sollen nur solche Daten genutzt werden, die entweder frei verfügbar sind oder mit dem Hochschulzugang zu beschaffen sind. Das schließt präparierte Datensätze von Bezahlseiten aus.\\
\hline
Risks and contingencies & \begin{itemize}
\item keinen Zugriff auf Azure ML Studio (Server-seitige Probleme; Ablauf der Free-Subscription)
\item benötigte Daten nicht verfügbar
\item \todo{todo}
\item\todo{todo}
\end{itemize} \\
\hline
Terminology & Als Glossar dient das Abkürzungsverzeichnis zu Beginn der Arbeit. \\
\hline
Costs and benefits & --- \\
\hline
\caption{Output des Schrittes "Assess the Situation"}
\end{longtable}


\subsection{Determine the Data Mining Goals}


\begin{longtable}[!h]{|p{4,5cm}|p{12cm}|}
\hline
\textbf{Output} & \textbf{Beschreibung} \\ 
\hhline{==}
Data mining goals & Es sollen mit Hilfe von Azure Machine Learning Studio mehrere Machine Learning Models erzeugt werden. Sie sollen folgende Sachverhalte \todo{Wort} vorhersagen:
\begin{itemize}
\item Steigt der Kurs innerhalb eines 24-Stunden Zyklus über das Hoch des vorhergehenden 24-Stunden Zyklus? (Mögliche Antworten sind $ja := 1$ und $nein := 0$) 
\item Fällt der Kurs innerhalb eines 24-Stunden Zyklus unter das Tief des vorhergehenden 24-Stunden Zyklus? (Mögliche Antworten sind $ja := 1$ und $nein := 0$)
\item Wie hoch ist das Hoch des nächsten 24-Stunden Zyklus? (Antwort: Preis eines Bitcoins/Ethers in USD)
\item Wie niedrig ist das Tief des nächsten 24-Stunden Zyklus? (Antwort: Preis eines Bitcoins/Ethers in USD)
\end{itemize} 
Bei den ersten beiden Fällen handelt es sich um eine two-class classification (siehe \ref{subsubsec:classification}). Die letzten beiden Fragen fallen in das Gebiet der Regressionen (siehe \ref{subsubsec:regression}). \\
\hline
Data mining success criteria & Die Klassifikationen werden bewertet anhand:
\begin{itemize}
\item Accuracy: $ \frac{t_{p} + t_{n}}{t_{p} + t_{n} + f_{p} + f_{n}}$ (Anteil der insgesamt richtig vorhergesagten Werte)
\item Precision: $ \frac{t_{p}}{t_{p} + f_{p}}$ (Anteil der richtig vorhergesagten true-Werte)
\item Recall: $ \frac{t_{p}}{t_{p} + f_{n}}$ (Anteil der richtig vorhergesagten true-Werte an allen eigentlich richtigen true-Werten)
\item F1-score: $ 2 \times \frac{Precision \times Recall}{Precision + Recall}$ (Harmonisches Mittel aus Precision und Recall; Bester Wert = 1)
\item AUC: Fläche unter der Receiver Operating Characteristics (ROC) Kurve; "Je besser die Klassifizierungsfähigkeit des Klassifikators desto höher ist der AUC-Wert"\citep[ROC-Kurve]{lohninger_grundlagen_2013} 
\end{itemize} 
$ t_{p} : true \quad positive $ \newline
$ t_{n} : true \quad negative $ \newline
$ f_{p} : false \quad positive $ \newline
$ f_{n} : false \quad negative $ \newline
\\
& Die Regressionen werden bewertet anhand:
\begin{itemize}
\item Mean absolute error (MAE): Unterschied von vorhergesagten und tatsächlichen Werten; je kleiner desto besser\citep{mircosoft_evaluate_2017}
\item Root mean squared error (RMSE): Macht eine Aussage darüber, "wie gut eine Funktionskurve an vorliegende Daten angepasst ist"; "je größer der RMSE [...], desto schlechter"\citep{statista_root_nodate}
\item Relative absolute error (RAE), Relative squared error (RSE): Ähnlich dem RMSE, aber für den Vergleich von Regressionen mit unterschiedlichen Maßeinheiten geeignet.\citep{dr._sayad_model_2017} Im vorliegenden Fall können die anderen Größen herangezogen werden, da alle Modelle die gleiche Einheit für die vorhergesagten Werte nutzen.
\item Coefficient of determination: Gütemaß für die Aussagekraft der Regression; 
\newline
$Regression \quad passt \quad perfekt := 1$ 
\newline 
$Regression \quad erklärt \quad nichts := 0$;
\newline 
kleine Werte sind normal, große  sollten misstrauisch machen\citep{mircosoft_evaluate_2017}
\end{itemize}
\\
\hline
\caption{Output des Schrittes "Determine the Data Mining Goals"}
\end{longtable}


Ziel: technische Beschreibung der Ziele
--> Find any Patterns, correlations, trends etc. in crypto data (1: de-/increase; 2: dollar exchange rate)

\subsection{Produce a Project Plan}
Ziel: Projektplan erstellen
--> Schritte, die durchgeführt werden (Gliederung hier ca)
--> Timeline (wie verältnis; evtl. auslassen?)
--> Risiken (gar nichts finden; Daten Fehlen; techn. Probleme mit Azure...)
--> Werkzeuge und Techniken (R, RStudio, Jupyter Notebooks, Azure ML, Excel)

\section{Data Understanding}
\subsection{Collect the Initial Data}
Ziel: daten beschaffen
--> oben, welche Sachen hätte ich gerne; hier dann richtige Daten und schon in azure laden!

\subsection{Describe the Data}
Ziel: Format, größe, Anzahl der Beobachtungen; Beschaffenheit
-->
--> sind die Daten okay? 

\subsection{Explore the Data}
Ziel: Einfache Analyse der Zielvariablen 
--> (steigend, fallend, preis?)

\subsection{Verify Data Quality}
Ziel: Qualität sichern; common sense
--> alle preise in usd; zeit in ttmmyyyy oder so
--> daten vollständig für die sachen?
--> kommas und punkte richtig?

\section{Data Preperation}
\subsection{Select Data}
Ziel: Welche Daten sind wichtig für die Analyse?
--> Vlt Daten wochenweise aggregieren
--> 

\subsection{Clean Data}
Ziel: Sachen aus "Verfiy Data Quality" aufgreifen;
--> Lücken füllen, Preise, Datum

\subsection{Construct Data}
Ziel: Daten so umbauen, dass sie in algorithmus passen
--> neue Zeilen hinzufügen; Verdoppeln w/e

\subsection{Integrate Data}
Ziel: Daten zusammenführen
--> die fertigen Datensätze, die ich gefunden habe in die BTC-, ETH-Kursdaten überführen

\subsection{Format Data}
evtl. auspassen

\section{Modeling}
\subsection{Select the Modeling Technique}
Ziel: Auswahl der Algorithmen
--> Supervised (Continous value --> wert; two-class (hoch/runter))

\subsection{Generate Test Design}
Ziel: Training und error rates betrachten
--> azure split data experiment item!

\subsection{Build the Model}
Ziel: Models bauen
--> "azure ml run"

\subsection{Assess the Model}


\section{Evaluation}
\subsection{Evaluate Results}
\subsection{Review Process}
\subsection{Determine Next Steps}


\section{Deployment}
\subsection{Plan Deployment}
\subsection{Plan Monitoring and Maintenance}
\subsection{Produce Final Report}
\subsection{Review Project}


