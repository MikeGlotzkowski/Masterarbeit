\chapter{Interpretation, Schlussbetrachtung und Fazit}\label{chap:Bewertung}


\section{Bewertung von Azure Machine Learning Studio}\label{sec:BeswertungAzure}
Bei Azure Machine Learning Studio handelt es sich um Werkzeug mit klarer Oberfläche. Das Einlernen und Navigieren zwischen Experimenten und Projekten ist sehr einfach. Die Funktionen sind auf dem ersten Blick verständlich. Es bietet die Standardoperationen des Supervised Learning an (Read, Clean, Transform, Split, Train, Tune, Score, Evaluate). Dadurch gelingt es, schnell zu Ergebnissen zu kommen.\newline
Das Werkzeug vermittelt den Eindruck, dass nicht viel Hintergrundwissen benötigt wird. Dies stimmt bei genauerem Hinsehen jedoch nicht. Bei der Auswahl der Algorithmen lassen sich alle Algorithmen auswählen, obwohl sie nicht auf das Problem passen. So lässt sich beispielsweise eine 'Two-Class Bayes Point Machine' auf ein offensichtlich nicht-lineares Problem anwenden. Dies ist kein Fehler des Werkzeugs, es zeigt nur auf, dass nicht auf Hintergrundwissen verzichtet werden kann. Dies Fällt auch auf, wenn  das Experiment Item 'Clean Missing Data' genutzt wird: Es muss zwischen acht Möglichen Bereinigungsmethoden entscheiden werden, zu denen auch Probabilistic PCA und MICE gehören. Allein diese Auswahl erfordert Recherchearbeit.
Andererseits bietet sich auch die Möglichkeit, verschiedene Einstellungen schnell und einfach auszuprobieren und so über 'trial und error' zu lernen.\newline
Ein großer Kritikpunkt ist, dass es passieren kann, dass Experimente beim Durchlaufen nicht terminieren. Sie laufen ewig oder werden nach einer Stunden ohne Fehlermeldung beendet. In der Free-Version ist eine parallele Verarbeitung mehrerer Experimente nicht möglich, deswegen führen diese Fehler zum Stillstand.\newline
Die Dokumentation ist angemessen und gibt einen guten Überblick über alle Features. An manchen Stellen wäre etwas mehr Information jedoch wünschenswert. Die Untersuchung ist an die Limitation von 100 Experiment Items pro Experiment und die maximale Experimentdauer von einer Stunde gestoßen.\newline
Azure Machine Learning bietet nur Supervised Learning an.\newline
Grundsätzlich ist der Erweiterbarkeit durch eigene Skripte (Python oder R) keine Grenzen gesetzt, es kann im Tool jedoch kein Debugging durchgeführt werden. Nicht-trivialer Code muss zuvor lokal getestet werden. Ebenfalls gibt es keine Versionsverwaltung, was für Analysten mit Software Engineering Hintergrund sicherlich ein Manko darstellt.\newline
Als Fazit kann festgehalten werden, dass Azure Machine Learning Studio ein gutes Werkzeug ist, das alle Grundfunktionen beinhaltet. Es ist gut dokumentiert und gut für Standardprobleme einsetzbar. Es lässt sich einerseits für Vorstudien in Projekten empfehlen, wenn herausgefunden werden muss, ob ein Problem mit den Mitteln des Machine Learning lösbar ist. Andererseits bietet es durch viele integrierte Beispiele einen perfekten Rahmen für das Lernen des Vorgehens beim Machine Learning.

\section{Bewertung des CRISP-DM Referenzmodells}\label{sec:BeswertungCrisp}
Das CRISP-DM Referenzmodell ist ein sehr guter Leitfragen für alle Beteiligten an einem Data Mining Projekt. Für allem für Analysten mit wenig Erfahrung bietet der User Guide eine gewaltige Hilfestellung. Das Modell legt viel Wert auf Dokumentation und Nachvollziehbarkeit. Es versucht, zwischen den betriebswirtschaftlichen, fachlichen und technischen Aspekten zu unterscheiden. Dies gelingt nicht immer ganz. Durch diese Betrachtung aus verschiedenen Blickwinkeln bietet der User Guide viele Tipps und stellt sicher, dass nichts vergessen oder übersehen wird.\newline
Empfehlenswert ist eine Anwendung des Modells als roter Faden in Data Mining, Machine Learning und anderen Data Science oder Analytics-Projekten. Es muss nur so abgeändert werden, dass es auf das Problem passt. Da es sehr umfangreich ist, kann - wie in der vorliegenden Arbeit - etwas weggelassen oder hinzugefügt werden.

\section{Bewertung der Ergebnisse des Machine Learning}\label{sec:BewertungML}
Der Analyseansatz in der Arbeit war, mit der Recherche nach Einflussfaktoren zu beginnen. Dabei wurden viele Faktoren ausgewählt. Diese wurden genauer untersucht, ausgedünnt und bearbeitet. Anschließend wurde das Machine Learning durchgeführt. Wenn dieser Ansatz verfolgt wird, ist es empfehlenswert, im Vorfeld bereits einen oder zwei Algorithmen auszuwählen (z.B. 'Two-Class Bayes Point Machine', da er widerstandsfähig gegen Overfitting ist), die genutzt werden sollen. Dadurch kann mehr Zeit aufgewendet werden, die Daten in eine bessere Struktur zu bringen.\newline
Mit dem gewonnene Wissen empfiehlt sich jedoch eher die Auswahl von einem oder zwei Faktoren (z.B. Kurs des Dow Jones (DJIA) und des Goldpreises). Dadurch können mehr Hilfsanalysen (z.B. Auseinanderschneiden (engl. slicing) der Daten und Suchen nach Zusammenhängen in bestimmten Zeitperioden) durchgeführt werden. Auch wird weniger Zeit für das Beschreiben und Zusammenführen der Daten verwendet als für detaillierte Untersuchungen.\newline
Es kann festgehalten werden, dass die Arbeit vorherige Untersuchungen unterstreicht, dass der Kursverlauf von Kryptowährungen nicht zufällig ist, sondern mit Faktoren wie dem öffentlichen Interesse (z.B. Google Suchanfragen) zusammenhängt.\newline
Eine andere Erkenntnis ist, dass die Kurse im Analysezeitraum tendenziell steigend. Aus diesem Grund besitzen Regressionen einen sehr hohen Wert für $ R^2 $. Eventuell muss eine andere Metrik hier zum Einsatz kommen.\newline
Im realen Betrieb eines Systems zur Vorhersage, müsste darauf geachtet werden, dass die, zur Analyse verwendeten Daten, parallel erhoben werden müssen. So kann sich beispielsweise die Größe "Tageshoch des WTI-Ölpreises" ständig ändern. Außerdem handelt es sich bei BTC und ETH um die wahrscheinlich prominentesten Kryptowährungen. Kleinere Altcoins unterliegen möglicherweise anderen Einflüssen. Auch die Wechselwirkung zwischen den Währungen darf nicht vernachlässigt werden.\newline
Zum Schluss muss noch erwähnt werden, dass es sich definitiv um ein sehr komplexes Problem handelt, das eventuell mit Standardvorgehen und -mitteln nicht lösbar ist.