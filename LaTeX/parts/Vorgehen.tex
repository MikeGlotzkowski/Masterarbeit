\chapter{Vorgehen und Ziele}\label{chapter:Vorgehen}
Nachdem in Kapitel \ref{chap:motivation} das Thema der Arbeit (\ref{sec:thema}) vorgestellt und die Motivation dahinter erläutert wurde (\ref{sec:cryptocurrency} bis \ref{sec:SaaS}), wird nun der restliche Aufbau vorgestellt.
Die nächsten vier Kapitel befassen sich mit den Grundlagen hinter der Arbeit (\textbf{Theorie}): 
\begin{itemize}
\item Kapitel \ref{sec:DataMiningFrameworks} geht auf Data Mining Frameworks ein. Dabei werden das Knowledge Discovery in Databases (KDD) process model (\ref{sec:kdd}) und der CRoss Industrial Standard Process for Data Mining (CRISP – DM (\ref{sec:crispdm}) betrachtet. Anschließend folgt eine Entscheidung für CRISP-DM in Punkt \ref{sec:crispdmdec}.
\item Daraufhin (Kapitel \ref{sec:MachineLearning}) folgen Erklärungen zu den Kategorien des Machine Learning: Supervised Learning (\ref{subsec:sl}), Unsupervised Learning (\ref{sec:us1}), Semi-supervised Learning (\ref{sec:ssl1}), Active Learning (\ref{sec:al1}) und Reinforcement Learning (\ref{sec:rl1}).
\item Genauere (technische) Details zu Kryptowährungen sind in Kapitel \ref{sec:cryptocurrency2} festgehalten. Hier sind Definitionen zu Begriffen zu finden, die später für die Analyse genutzt werden.
\item Den Abschluss des Theorieteils stellt die Allgemeine Beschreibung (\ref{sec:ab1}), der Aufbau und die Komponenten (\ref{sec:auK1}) von Microsoft Azure Machine Learning Studio in Kapitel \ref{sec:msmls} dar.
\end{itemize}
Im \textbf{Praxisteil} in Kapitel \ref{chap:praxis} werden
\begin{itemize}
\item zuerst Ziele und Projektressourcen festgelegt (\ref{sec:p1}).
\item Daraufhin werden Daten beschafft, beschrieben und untersucht (\ref{sec:p2}).
\item Nach ihrer Auswahl und Aufbereitung (\ref{sec:p3}) folgt
\item die Auswahl der Machine Learning Algorithmen, die Festlegung des Testdesigns, Durchführen des Machine Learnings und die Auflistung der Ergebnisse (\ref{sec:p4}).
\item Die Ergebnisbewertung, der Prozessrückblick und die nächsten Schritte (\ref{sec:p4}) stellen den Abschluss dar.
\end{itemize}
Das Schlusskapitel (\ref{chap:Bewertung}) bewertet die drei Kernelemente der Arbeit:
\begin{itemize}
\item das Azure Machine Learning Studio (\ref{sec:BeswertungAzure}),
\item das CRISP-DM Referenzmodell (\ref{sec:BeswertungCrisp}) und
\item die Ergebnisse des Machine Learning (\ref{sec:BewertungML}).
\end{itemize}
Das Hauptziel der Arbeit ist es, zu untersuchen, ob sich der Kurs von Kryptowährungen mit Hilfe von Machine Learning vorauszusagen lässt oder nicht.\par
Für alle Teile sei angemerkt, dass die Fachliteratur fast ausschließlich in englischer Sprache verfügbar ist. In dieser Arbeit wird deshalb die Entscheidung getroffen, die Fachbegriffe nicht zu übersetzten. Auch die Prozessschritte und -artefakte (Outputs) des CRISP-DM-Referenzmodells sind davon betroffen. 
