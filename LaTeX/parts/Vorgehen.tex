\chapter{Vorgehen und Ziele}\label{chapter:Vorgehen}
Nach der Einführung in das Thema und dem Einordnen in aktuelle Themenfelder, wird nun das Vorgehen und das Ziel der Arbeit erläutert.\newline
Der anschließende Abschnitt xxx befasst sich mit den Grundlagen, die für das Verständnis der Ausarbeitung nötig sind. Dort wird beispielsweise auf die verschiedenen Kategorien des Machine Learning (in xxx) und die zugehörigen Algorithmen und Verfahren eingegangen. Der nachfolgende Teil xxx befasst sich damit, Einflussfaktoren auf die Kurse von Kryptowährungen zu isolieren. Sind die Einflüsse gefunden, wird dargelegt, wie diese als Daten(satz) abgebildet werden können (xxx) und was als Quelle der Daten dient (xxx). In Punkt xxx werden die Datensätze beschrieben. Anschließend (Gliederungspunkt xxx) wird gezeigt, wie die Analyse durchgeführt wird. Dabei wird er Prozess yyy (siehe xxx) \todo{welcher Prozess? CRISP/KDD...; bereinigung etc..} durchlaufen. In Abschnitt xxx werden die Ergebnisse interpretiert und es werden Schlüsse gezogen. \todo{bessere Formulierung} Den Abschluss stellt der Ausblick (xxx) dar. Dieser Teil befasst sich damit, welchen Nutzen die Arbeit bringt (xxx) und wie die Erkenntnisse weiter verwendet werden können (xxx).\newline
\todo{refs} \newline
\todo{related work}\newline
\todo{später genauer eingehen auf die Sachen}
Als Ziel steht über der Arbeit, ob es möglich ist, den Kurs oder Kursschwankungen von Kryptowährungen mit Hilfe von Machine Learning vorauszusagen oder nicht. \todo{braucht man das "oder nicht"?}\newline
\todo{grafische Darstellung anfügen}
